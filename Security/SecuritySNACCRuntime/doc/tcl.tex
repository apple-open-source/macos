% file: .../doc/tcl.tex

% $Header: /cvs/root/Security/SecuritySNACCRuntime/doc/Attic/tcl.tex,v 1.1.1.1 2001/05/18 23:14:10 mb Exp $
% $Log: tcl.tex,v $
% Revision 1.1.1.1  2001/05/18 23:14:10  mb
% Move from private repository to open source repository
%
% Revision 1.1.1.1  1999/03/16 18:05:54  aram
% Originals from SMIME Free Library.
%
% Revision 1.1  1997/01/01 22:47:40  rj
% first check-in
%

\chapter{\label{tcl-if-chapter}Tcl Interface}

\section{\label{intro-tcl-section}Introduction}

This chapter describes the Snacc's Tcl interface, or: the metacode's link to the outside world.

Tcl is a simple scripting language which the author, John K. Ousterhout, describes in his book titled ``Tcl and the TK Toolkit'' \cite{tcl-book}.
Tcl's purpose is to be embedded into other applications, to provide a user interface by extending the language.
Tk, an implementation of the Motif look and feel, is the first and best known extension to Tcl and is described in the same book.

Tcl has got only one data type, the NUL terminated character string.
Tcl supports other data types like integers and lists, but they are represented as strings.
A function operating on an integer first converts the string into an integer, performs its operation, converts the resulting value back into another string and returns it to the Tcl interpreter.
Since lists and even the Tcl procedures are kept as strings, Tcl is rather slow.
Computations in Tcl should best be kept at a minimum, and all intensive work should be wrapped into C or C++ functions and be made available as Tcl commands.

Since procedures and bodies of loops are kept in string form and parsed for every invocation, comments should be put outside code that is executed \emph{very} often.

From Tcl's point of view, Snacc's Tcl interface is nothing but yet another Tcl extension.
The Snacc Tcl interface extends the Tcl language by only one command, {\Tcl snacc}.
The first argument to this command specifies the action to be taken.
This method is very practical for combining Tcl extensions since it avoids collisions with new command names from other extensions.
For example, the Tcl core defines an {\Tcl open} command.
Snacc's Tcl interface wants to offer one as well and has to choose another name.
This could have been done by naming it {\Tcl snacc\_open}, but I think it is better to stick to Tcl's well established convention and so the Tcl interface's open command became {\Tcl snacc open}.
To simplify the wording, I will refer to the `snacc subcommands' simply as `commands'.

The usual (non-metacode) snacc generated functions operate on memory buffers containing BER encoded data; they convert them into hierarchical C++ data structures and vice versa.

The Tcl interface is designed to allow controlled fine grained access to this hierarchical C++ data structure, to read and modify its contents.
While both the C++ code and the Tcl look very similar, for example\dots\\[1ex]
{\C
// this is C++ code\\
x-->foo-->bar = 42;\\[1ex]
}
\dots\ and\dots\\[1ex]
{\Tcl
\# this is Tcl code\\
snacc set \{x foo bar\} 42
}

\dots\ the C++ code gets compiled and the identifiers get turned into pointers and numeric offsets, and the Tcl code gets interpreted and has to mimic the C++ compiler at run time.
This is what the metacode from chapter~\ref{meta-chapter} is for.

To enable snacc's Tcl code generator, you have to give an additional {\ufn -tcl} option, followed by the list of PDU types.
The {\ufn -meta} option can (and should) be omitted.

\section{The {\Tcl snacc} Tcl command}

This section explains the Tcl (sub)commands provided by the Snacc extension.
The commands are grouped in three catagories, commands operating on files (both their external and internal representation), commands accessing the meta information and commands operating on the content itself.

The file commands check the return value from system calls and behave like for example the Tcl {\Tcl open} command, that is, they set the {\Tcl errorCode} variable to {\Tcl POSIX \emph{errno}}, e.g. {\Tcl POSIX ENOENT \{No such file or directory\}}.

The code should be fairly robust, not just against user and programmer errors from `outside' (using the {\Tcl snacc} Tcl command), but against errors from the `inside' as well such as illegal numeric values for enumeration types or illegal choice settings as well.

There are two types of errors:
\begin{enumerate}
  \item programmer errors, where the program has no other choice as to print a regret to the user and exit
  \item user errors, such as trying to write to a read-only file, where the program should tell the user about their mistake and let them try something else.
\end{enumerate}
The Tcl interface code helps the programmer for the second type of error by setting Tcl's {\Tcl errorCode} variable.
The program can {\Tcl catch} any error, and, based on the {\Tcl errorCode}, choose to deal with the mistake or rethrow the error that it is not prepared to handle.

\subsection{File commands}

Most snacc Tcl commands operate on so-called files.
A file is an internal data structure that
\begin{itemize}
  \item references the C++ representation of an ASN.1 data structure as a pointer to {\C AsnType}
  \item may be associated with an external file in the file system
\end{itemize}

The commands operating on these files are as follows:

\begin{description}%{
  \item[{\Tcl snacc create \emph{type}}]
    The command creates a file consisting only of an instance of type \emph{type}.
    \emph{type} has to be denoted as one argument, a Tcl list with two elements, module and type.
    No external filename is associated with this file.
%    The command returns a file handle that consists of letters and digits only and may therefore be used to construct a component in the Tk widget tree.

  \item[{\Tcl snacc open \emph{type filename} ?\emph{flags}? ?\emph{mode}?}]
    Open a file and read and decode its contents.
    \emph{type} has to be denoted as one argument, a list with two elements, module and type.
    The optional \emph{flags} may consist of:
    \begin{description}
      \item[{\Tcl create}] If the file does not exist, create it.
	If this flag is not given and the file does not already exist, an error occurs.
      \item[{\Tcl truncate}] If the file exists, drop its contents.
      \item[{\Tcl \emph{access}}] which may be either {\Tcl ro} or {\Tcl rw}, denoting read only and read/write access.
	If no access mode is specified, the file will be opened read/write if it is writable, and read only otherwise.
    \end{description}
    If the file is created, its mode is set to \emph{mode}, minus umask, of course.
    \emph{mode} may be any value accepted by {\C Tcl\_GetInt(3)} (the function accepts octal values).
    At last, if the file could be opened, its contents is read and BER decoded.
    As for {\Tcl snacc create} above, a file handle is returned.

    If the file cannot be opened, an error is returned identical to Tcl's {\Tcl open} command.

    More errors can be returned, as described under {\Tcl snacc read} below.

  \item[{\Tcl snacc close \emph{file}}]
    closes the file \emph{file} and invalidates the file handle.

  \item[{\Tcl snacc read \emph{file} ?\emph{type filename}?}]
    without the \emph{filename}, rereads the file from its old place; otherwise opens \emph{filename}, reads its contents into \emph{file} and closes it.
    The file's contents gets BER decoded.

    In case no \emph{filename} has been given but the \emph{file} is not associated with a filename, an error is returned and {\Tcl errorCode} is set to {\Tcl SNACC MUSTOPEN}.

    If Snacc's decoding routines detect an error, a Tcl error is returned and {\Tcl errorCode} is set to {\Tcl SNACC DECODE \emph{errval}} where \emph{errval} is the value returned by {\C setjmp()} (see sections~\ref{error-C++-section} and~\ref{lib-err-C-section} on pages~\pageref{error-C++-section} and~\pageref{lib-err-C-section}, respectively).

    If the input file is too short, the buffer will signal a read error and a Tcl error will be returned, with {\Tcl errorCode} set to {\Tcl SNACC DECODE EOBUF}.

  \item[{\Tcl snacc write \emph{file} ?\emph{filename}?}]
    BER encodes the file, then writes the file to its old place in case no \emph{filename} has been given, or opens \emph{filename}, writes \emph{file} into it and closes it.

    In case no \emph{filename} has been given but the \emph{file} is not associated with a filename, an error is returned and {\Tcl errorCode} is set to {\Tcl SNACC MUSTOPEN}.
    If you try to write to a read-only file, an error is returned and {\Tcl errorCode} is set to {\Tcl SNACC WRITE READONLY}.

  \item[{\Tcl snacc finfo \emph{file}}]
    returns a list with two elements, the file name associated with it (the empty string if no external file name is associated with it) and an identifier which may be
    \begin{description}%{
      \item[{\Tcl bad}] the file is not associated with an external file.
      \item[{\Tcl rw}] the external file has been opened read/write.
      \item[{\Tcl ro}] the external file has been opened read only.
    \end{description}%}
\end{description}%}

Since Tcl cannot operate on binary strings (that is, strings containing NUL bytes), but ASN.1 octet strings may contain arbitrary binary data, the binary data has to be converted into a replacement notation that Tcl can work with and that can be converted back to binary without loss of information.
The conversion I chose is fairly simple: NUL is converted into a backslash followed by a zero digit, and every backslash is doubled.

These conversions for the most part take place automatically.
In fact, there is only one point where the binary representation is necessary, when you want to read or write data from or into a file on disk.
Two functions have been written to offer this: the export function converts and writes an octet string to an external file, and the import function reads binary data from a file and converts it to the Tcl compatible representation.
Unlike the functions described above, these two do not operate on ASN.1 files, that is, the contents is not BER decoded/encoded, but may be used for any file in the file system.

\begin{description}%{
  \item[{\Tcl snacc import \emph{filename}}] opens the file named, reads its contents, closes it, performs the above described conversion and returns the resulting Tcl string.
  \item[{\Tcl snacc export \emph{string filename}}] converts the Tcl string into its binary counterpart, opens the file named, writes the binary buffer into it and closes it.
    The file is created and truncated as necessary.
    The command returns the empty string.
\end{description}%}

\subsection{Generic Information Retrieval}

The following functions return information about the modules and their types.
(This information is independent of any file instance, it is the information from the type descriptions in the {\ufn .asn1} files.)

\begin{description}%{
  \item[{\Tcl snacc modules}]
    returns a list of module identifiers.
  \item[{\Tcl snacc types ?\emph{module}?}]
    if a \emph{module} is specified, returns a list of all type names of that module.
    otherwise, a list of all types is returned as a list of pairs, where each pair consists of the module name and the type name.
  \item[{\Tcl snacc type \emph{type}}]
    where \emph{type} is a list with two elements, module and type.
    This command returns a list with the following four elements:
    \begin{enumerate}%{
    \setcounter{enumi}{-1}
      \item the content type as a list consisting of module name and type name
      \item an identifier that is either {\Tcl pdu} or {\Tcl sub} depending on the list of PDUs that had been given after snacc's {\ufn -tcl} option.
      \item the ASN.1 type (e.g. INTEGER or CHOICE)
      \item a list of items that depends on the ASN.1 type:
      \begin{description}%{
	\item[INTEGER] a (possibly empty) list of pairs of name and value for each named value.
	\item[ENUMERATED] a (non-empty) list of names.
	\item[SET, SEQUENCE \textnormal{and} CHOICE] a list of lists of four elements similar to that being described here.
	  Element~0 is the subtypes name, then follow content type (a pair consisting of module name and type name), \emph{pdu} vs. \emph{sub} and finally the ASN.1 type.
	  (The fourth element of the outer list is omitted for obvious reasons: it would explode the type's description.)
      \end{description}%}
    \end{enumerate}%}
\end{description}%}

\subsection{Operations on Content and Structure}

Finally, the last last four functions operate on the file instances itself.
All four commands get a \emph{path} argument that is constructed as follows:
\begin{itemize}
  \item Every \emph{path} starts with a file handle as returned by {\Tcl snacc create} or {\Tcl snacc open}.
  \item All subsequent path elements, except for the last, must indicate elements of composed types.
    For CHOICE, SET and SEQUENCE, these are member names, for SET OF and SEQUENCE OF, these are numeric indices.
  \item The last path element may reference a simple type.
  \item For SET OF and SEQUENCE OF, instead of a numeric index, a pair consisting of the word {\Tcl insert} followed by a numeric index may be specified.
    In this case, a new list element is inserted before that addressed by the index.
    The index must be in the range $0\ldots{}n-1$ to address existing elements and it must be in the range $0\ldots{}n$ for insertion, where in both cases $n$ is the number of elements in the list.
  \item For {\Tcl snacc unset}, the path must point to an optional member of a SET or SEQUENCE or to an element of a SET OF or SEQUENCE OF.
\end{itemize}

The commands are:
\begin{description}%{
  \item[{\Tcl snacc info \emph{path}}] returns information about the value pointed to by \emph{path}.
    The information returned is quite similar to that of {\Tcl snacc type} above, with the following exceptions:
    \begin{itemize}%{
      \item element~0, the content type, contains empty names for types that have not been given a name (e.g. a SET member of type OCTET STRING
	Example: the {\ASN contents} member in type {\ASN File} in file {\ufn edex1.asn1} (page~\pageref{edex1.asn1}) {\Tcl snacc info} returns {\Tcl \{\{\} \{\}\} sub \{OCTET STRING\}}).
      \item the number of elements depends on the ASN.1 type:
	\begin{description}%{
	  \item simple types (\textbf{NULL}, \textbf{BOOLEAN}, \textbf{INTEGER}, \textbf{ENUMERATED}, \textbf{REAL}, \textbf{BIT STRING} and \textbf{OCTET STRING}):
	    no additional elements are returned.
	    For the list of named values for INTEGER, ENUMERATED and BIT STRING, you have to call {\Tcl snacc type [lindex [snacc info \emph{path}]~0]}, unless the content type equals {\Tcl \{\{\}~\{\}\}}.
	  \item[CHOICE]
	    a total of five elements is returned, number~3 is the name of the choice member currently chosen, and the final element number~4 is an identifier that is either {\Tcl void} or {\Tcl valid} depending on whether the pointer representing the choice member is {\C NULL} or pointing to some data.
	  \item[SET \textnormal{and} SEQUENCE]
	    a fourth element, a list of pairs, is returned, where the pairs are built from the member name and an identifier that is either {\Tcl valid} or {\Tcl void}
	  \item[SET OF \textnormal{and} SEQUENCE OF]
	    the number of items is returned as element number~3.
	\end{description}%}
    \end{itemize}%}
  \item[{\Tcl snacc get \emph{path}}]
    returns the value of the subtree pointed to by \emph{path}.
    The value returned is a simple string for simple types, and a hierarchical structure (in Tcl that is a list of lists) otherwise.
    \begin{description}%{
      \item[NULL]
	the empty string is returned.
      \item[BOOLEAN]
	the value is returned as {\Tcl TRUE} or {\Tcl FALSE}.
      \item[INTEGER]
	the numeric value is returned, even if it has been assigned a name.
      \item[ENUMERATED]
	the symbolic value is returned.
	The numeric values are inaccessible through the Tcl interface.
	If the object happens to contain an illegal numeric value, an error is returned and {\Tcl errorCode} is set to {\Tcl SNACC ILLENUM}.
      \item[REAL]
	the value is returned as formatted by {\C sprintf (\dots, "\%g", \dots)}, except for the special values {\ASN PLUS-INFINITY} and {\ASN MINUS-INFINITY} which are returned as {\Tcl +inf} and {\Tcl -inf}, respectively.
      \item[BIT STRING]
	a string, consisting solely of `0' and `1', is returned.
      \item[OCTET STRING]
	the binary string is returned as is, except for the unavoidable NUL-escape described above.
      \item[OBJECT IDENTIFIER]
	the value is returned as a list of numbers.
      \item[CHOICE]
	is returned as a pair, the choice member chosen and its value.
      \item[SET \textnormal{and} SEQUENCE]
	are returned as a list of pairs of member name and value.
	Absent OPTIONAL members are left out from the list.
      \item[SET OF \textnormal{and} SEQUENCE OF]
	are returned as a list of values.
    \end{description}%}
  \item[{\Tcl snacc set \emph{path value}}] sets the subtree identified by \emph{path} to \emph{value}.
    The value must be of the form 
    \begin{description}%{
      \item[NULL]
	the only legal value is the empty string.
	otherwise, an error is returned and {\Tcl errorCode} is set to {\Tcl SNACC ILLNULL}.
      \item[BOOLEAN]
	any value that is accepted by {\C Tcl\_GetBoolean}(3) is fine.
      \item[INTEGER]
	both the numeric (as accepted by {\C Tcl\_GetInt(3)}) and the symbolic values are allowed.
      \item[ENUMERATED]
	any value must be specified by its name.
	If an illegal name is given, an error is returned and {\Tcl errorCode} is set to {\Tcl SNACC ILLENUM}.
      \item[REAL]
	the special values {\ASN PLUS-INFINITY} and {\ASN MINUS-INFINITY} have to be given as {\Tcl +inf} and {\Tcl -inf}, respectively.
	All other values may be specified in any format accepted by {\C Tcl\_GetDouble}(3).
      \item[BIT STRING]
	a string that must consist of `0' and `1' only has to be given.
	otherwise, an error is returned and {\Tcl errorCode} is set to {\Tcl SNACC ILLBIT}.
      \item[OCTET STRING]
	due to the NUL-escapes necessary, any string where a backslash is followed by either another backslash or a `0' digit is legal.
	Improper use of the escape character leads to an error and {\Tcl errorCode} will be set to {\Tcl SNACC ILLESC}.
      \item[OBJECT IDENTIFIER]
	the value has to be specified as a list of numbers.
	if the arc has less than 2 or more than 10 elements, an error is returned and {\Tcl errorCode} is set to {\Tcl SNACC ILLARC <2} or {\Tcl SNACC ILLARC >10}, respectively.
      \item[CHOICE]
	the value expected is a pair, the choice member chosen and its value.
	if an illegal member is specified, an error is returned and {\Tcl errorCode} is set to {\Tcl SNACC ILLCHOICE}.
      \item[SET \textnormal{and} SEQUENCE]
	the value has got to be a list of pairs of member name and value.
	Any member may be specified at most once.
	All mandatory members must be present.
	Failure to do so will result in an error and {\Tcl errorCode} to be set to {\Tcl SNACC DUPMEMB} or {\Tcl SNACC MISSMAND}, respectively.
	All optional members not listed in the value will be deallocated.
      \item[SET OF \textnormal{and} SEQUENCE OF]
	the whole list is replaced with the specified value that has to be a proper Tcl list.
    \end{description}%}
  \item[{\Tcl snacc unset \emph{path}}]
    unsets the subtree pointed to by \emph{path}.
    Only OPTIONAL members of SET and SEQUENCE types and list elements of SEQ OF and SEQUENCE OF may be unset.
    If you try to unset a mandatory SET or SEQUENCE member, an error is returned and {\Tcl errorCode} is set to {\Tcl SNACC MANDMEMB}.
\end{description}%}

I did not follow Tk's example where one has to set widget commands to {\Tcl \{\}} to delete them.
This method would have the drawback that one could not distinguish between an empty and a non-existing octet string (in C that would be {\C ""} vs. {\C NULL}).

The value returned by {\Tcl snacc get} may be very long, {\Tcl snacc get file0} returns the contents of the whole file!

\section{\label{snacc-examples}Examples}

The following example session shall illustrate the {\Tcl snacc} commands usage.
It assumes that the editor example files {\ufn edex0.asn1} and {\ufn edex1.asn1} (see appendix~\ref{edex-files} on page~\pageref{edex-files}) have been compiled into a binary that has been linked with the necessary libraries.

The notation used is as in the Tcl book \cite{tcl-book}, i.e. `$\Rightarrow$' indicates a normal return value and `$\varnothing$' indicates an error with the error message set in {\Tcl \emph{oblique typeface}}.

A look at the types available:

\begin{Tclex}
   & snacc types\\
\R & \{EdEx-Simple Hand\} \{EdEx-Structured StructuredChoice\} \{EdEx-Structured Coordinate\} \{EdEx-Structured CoordinateSeq\} \{EdEx-Structured RGBColor\} \{EdEx-Structured Simple\} \{EdEx-Simple File\} \{EdEx-Simple RainbowColor\} \{EdEx-Structured DirectorySetOf\} \{EdEx-Structured Various\} \{EdEx-Structured File1\} \{EdEx-Structured CoordinateSeq1\} \{EdEx-Structured Directory\} \{EdEx-Structured Structured\} \{EdEx-Simple DayOfTheWeek\}
\end{Tclex}

Create a file (without filename):

\begin{Tclex}
   & set file [snacc create \{EdEx-Structured Structured\}]\\
\R & file0
\end{Tclex}

The string returned is the file handle. It is used as the first snaccpath component in successive calls.

Look at the file's type:

\begin{Tclex}
   & snacc info \$file\\
\R & \{EdEx-Structured Structured\} sub SET \{\{coord valid\} \{color valid\}\}
\end{Tclex}

The file's type is a SET with the name `Structured' in module `EdEx-Structured' (it is defined in file {\ufn edex1.asn1} (see page~\pageref{edex1.asn1})).
The `sub' tells us that the type has not been marked as a PDU.
The SET has the components `coord' and `color', both are present (they are not OPTIONAL, i.e. mandatory).

Look at a component's type:

\begin{Tclex}
   & snacc info "\$file color"\\
\R & \{EdEx-Structured StructuredChoice\} sub CHOICE rainbow valid
\end{Tclex}

Snacc has generated the type name `StructuredChoice' for this type, this name was not defined in the {\ufn .asn1} file.
The CHOICE object currently is set to `rainbow'.
A CHOICE component is always present (CHOICE components may not be OPTIONAL), the `valid' is just for completeness.

Ask for the CHOICE's generic type information:

\begin{Tclex}
   & snacc type \{EdEx-Structured StructuredChoice\}\\
\R & \{EdEx-Structured StructuredChoice\} sub CHOICE \{\{rainbow \{EdEx-Simple RainbowColor\} sub INTEGER\} \{rgb \{EdEx-Structured RGBColor\} sub SEQUENCE\}\}
\end{Tclex}

The CHOICE type has two possible components, `rainbow', an INTEGER and `rgb', a SEQUENCE.

Look at the INTEGER's type information:

\begin{Tclex}
   & snacc type \{EdEx-Simple RainbowColor\}\\
\R & \{EdEx-Simple RainbowColor\} sub INTEGER \{\{red~0\} \{orange~1\} \{yellow~2\} \{green~3\} \{blue~4\} \{indigo~5\} \{violet~6\}\}
\end{Tclex}

The type has got named values.

Access the file contents:

\begin{Tclex}
   & snacc get \$file\\
\R & \{coord \{cartesian \{\{x 0\} \{y 0\}\}\}\} \{color \{rainbow 977768\}\}
\end{Tclex}

The color component contains garbage.
Change that:

\begin{Tclex}
   & snacc set "\$file color rainbow" green\\
\R &\\
   & snacc get "\$file color"\\
\R & rainbow 3
\end{Tclex}

Change it again, select the CHOICE's other component type, `rgb', and set its `red' component:

\begin{Tclex}
   & snacc set "\$file color rgb red" 256\\
\R &
\end{Tclex}

Changing a CHOICE component selection work only for write access, on read access this is not possible:

\begin{Tclex}
   & snacc get "\$file color rainbow"\\
\E & \emph{snacc get: illegal component "rainbow" in path}\\
   & snacc get "\$file color rgb"\\
\R & \{red 256\} \{green 544501616\} \{blue 1814045815\}
\end{Tclex}

Upon setting a SET or SEQUENCE type, all mandatory members have to be specified:

\begin{Tclex}
   & snacc set "\$file color rgb" \{\{green 0\} \{blue 0\}\}\\
\E & \emph{mandatory member "red" is missing in list}\\
   & snacc set "\$file color rgb" \{\{red 0\} \{green 256\} \{blue 0\}\}\\
\R &\\
   & snacc get "\$file color"\\
\R & rgb \{\{red 0\} \{green 256\} \{blue 0\}\}
\end{Tclex}

Finish up:

\begin{Tclex}
   & snacc close \$file\\
\R &\\
   & snacc get \$file\\
\E & \emph{snacc get: no file named "file0"}
\end{Tclex}

\section{\label{tcl-if-impl}Implementation}

The Tcl interface is implemented in {\ufn \dots/c++-lib/inc/tcl-if.h} and {\ufn \dots/c++-lib/src/tcl-if.C}.
It gets initialized with the help of {\ufn \dots/c++-lib/inc/init.h} and {\ufn \dots/c++-lib/src/tkAppInit.c}.

The {\Tcl snacc} commands implementation is pretty straight forward: check the arguments, call a metacode function to perform an action and return the result, which may indicate success or an error.

Care has been taken to check the return codes of all system calls and to set Tcl's {\Tcl errorCode} variable in case any system call returns an error.

The file {\ufn tkAppInit.c} contains the function that introduces the {\Tcl snacc} Tcl command to the Tcl interpreter.
The path that leads to the function's invocation is a little tricky and is described in section~\ref{editor-building}, ``Building Your Own Editor''.

\section{Setup for the Tcl Code Generator}

To compile Snacc with the Tcl interface code generator, you have got to fulfill the following conditions:
\begin{itemize}
  \item the configure script must be able to find {\ufn tclsh} and the Tcl/Tk libraries
  \item the preprocessor switches {\C NO\_META} and {\C NO\_TCL} in {\ufn \dots/policy.h} must not be set
\end{itemize}

\section{\label{tcl-if-deficiencies}Deficiencies}

\begin{itemize}
  \item
    Values defined in the ASN.1 files currently are inaccessible.
    Adding access functions to the metacode and Tcl interface is rather trivial: build an array of elements that hold a variable's name as a character string and an {\C AsnType~*} that points to the C++ variable.
    {\C a[i].val-->\_getdesc()} would return a pointer to the variable's type description.

    (First you should fix snacc's value parser as currently it lets some values silently vanish, for example the {\ASN victory} in {\ufn edex0.asn1} that you can find in appendix~\ref{edex0.asn1} on page~\pageref{edex0.asn1}.)
  \item
    The Tcl interface does not provide symbolic object identifiers.
    Mapping numeric to symbolic oids is a task that is difficult to get right since snacc translates\\
    {\ASN anOidVal OBJECT IDENTIFIER ::= \{ 1 2 foo(3) \}}\\
    and\\
    {\ASN anOidVal OBJECT IDENTIFIER ::= \{ 1 2 3 \}\\
    foo INTEGER ::= 3}\\
    into identical C++ code, but translating the second {\ASN anOidVal} into {\ASN \{ 1 2 foo \}} may in fact violate {\ASN foo}'s semantics.
\end{itemize}
